\textbf{Метод 1 --- LUP--разложение}\\

\textbf{Задание}

Реализовать алгоритм LUP--разложения матриц (с выбором главного элемента) в виде программы. Используя разработанное программное обеспечение, решить систему линейных алгебраических уравнений (СЛАУ). Для матрицы СЛАУ вычислить определитель и обратную матрицу.\\

\textbf{Вариант:} 3

$
\begin{cases}
9x_1-5x_2-6x_3+3x_4=-8\\
x_1-7x_2+x_3=38\\
3x_1-4x_2+9x_3=47\\
6x_1-x_2+9x_3+8x_4=-8
\end{cases}
$
\vspace{0.5cm}

\textbf{Описание алгоритма}

LUP--разложение матрицы $A$ представляет собой разложение матрицы  $A$ в произведение нижней и верхней треугольных матриц, а также матрицы перестановок $P$, т.е.

$$
PA=LU
$$

где $L$ --- нижняя треугольная матрица (матрица, у которой все элементы, находящиеся выше главной диагонали, равны нулю, $l_{ij}=0$ при $i<j$), $U$ --- верхняя треугольная матрица (матрица, у которой все элементы, находящиеся ниже главной диагонали, равны нулю, $u_{ij}=0$ при $i>j$).\\

LUP--разложение может быть построено с использованием метода Гаусса, используя формулу ниже для обнуления поддиагональных элементов: 
$$
a_{ij}^{(k)}=a_{ij}^{(k-1)}-\mu_i^{(k)}a_{kj}^{(k-1)}, \mu_i^{(k)}=\frac{a_{ik}^{(k-1)}}{a_{kk}^{(k-1)}}, i=\overline{k+1,n}, j=\overline{k, n}
$$

На первом этапе решается СЛАУ $Lz=b$. Поскольку матрица системы --- нижняя треугольная, решение можно записать в явном виде:

$$
z_1=b_1, z_i=b_i-\sum_{j=1}^{i-1}l_{ij}z_j, i=\overline{2,n}
$$

На втором этапе решается СЛАУ $Ux=z$ с верхней треугольной матрицей. Здесь, как и на предыдущем этапе, решение представляется в явном виде:

$$
x_n=\frac{z_n}{u_{nn}}, x_i=\frac{1}{u_{ii}}(z_i-\sum_{j=i+1}^{n}u_{ij}x_j), i=\overline{n-1,1}
$$\\

Отметим, что второй этап эквивалентен обратному ходу метода Гаусса, тогда как первый соответствует преобразованию правой части СЛАУ в процессе прямого хода.\\

В результате прямого хода метода Гаусса можно вычислить определитель матрицы $A$ исходной СЛАУ:

$$
det A=(-1)^pa_{11}a_{22}^1a_{33}^2 \cdot ... \cdot a_{nn}^{n-1}
$$

где $p$ --- число перестановок строк в процессе прямого хода, учитываются соответствующие перемены знаков вследствие перестановок строк.\\

\textbf{Реализация}

\begin{lstlisting}
TVector TSolve::_solveAx_is_b(const TMatrix& mL, 
                            const TMatrix& mU, 
                            const TVector& vB) {
    int tmpSz = vB.GetSize();    
	TVector z(tmpSz);
    TVector x(tmpSz);
	z[0] = vB[0];	
    for (int i = 1; i < tmpSz; ++i) {
        double tmpVal = 0.0;
        for (int j = 0; j < i; ++j)
            tmpVal += mL[i][j] * z[j];
        z[i] = vB[i] - tmpVal;
    }    
    x[tmpSz - 1] = z[tmpSz - 1] / mU[tmpSz - 1][tmpSz - 1];    
    for (int i = tmpSz - 1; i >= 0; --i) {
        double tmpVal = 0.0;
        for (int j = i + 1; j < tmpSz; ++j) {
            tmpVal += mU[i][j] * x[j];
        }
        x[i] = (z[i] - tmpVal) / mU[i][i];
    }
    z.Clear();
    return x;
}

int TSolve::ToSolveByGauss() {                 
    TFRFF* tmpRead = _readFromFile(pathFrom, Gauss);
    if (tmpRead == NULL)
        return -1;
    _matrA.SetLink(tmpRead->matr);
    _vecB.SetLink(tmpRead->vec);
    ofstream log("solve1Gauss.log", ios::out);
    log << "|Method Gauss (LUP)| by Alexander Bales 80-308" << endl << endl;    
    TMatrix L(_matrA.GetSizeRow(), _matrA.GetSizeCol(), Identity); 
    TMatrix U(_matrA);           
    int cntSwitchRowsAndColumns = 0;
    int posPrecc = -1;
    int tmpSz = min(U.GetSizeRow(), U.GetSizeCol());    
    for (int i = 0; i < tmpSz; ++i) {
        int posMax = U.FindPosMaxElemInColumn(i);           
        if (tmpSz - 1 != i) {
            U.SwapRows(posMax, i);            
            _matrA.SwapRows(posMax, i);
            _vecB.Swap(posMax, i); 
            if (posMax != i)
                cntSwitchRowsAndColumns++;                
            if (posPrecc != -1) {
                L.SwapRows(posMax, i);
                L.SwapColumns(posMax, i);                            
            }               
        }
                                
        for (int j = i + 1; j < tmpSz; ++j) {
            double koef = - U[j][i] / U[i][i];
            U[j][i] = 0.0;
            for (int k = i + 1; k < tmpSz; ++k) {
                U[j][k] = U[j][k] + koef * U[i][k];
            }
            L[j][i] = -koef;
        }
        posPrecc = posMax;
    }

    _vecX.SetLink(_solveAx_is_b(L, U, _vecB));	    
    L.Print(log, "L");
    U.Print(log, "U");    
    output << "det(A) = ";
    double detA = 1.0;
    for (int i = 0; i < tmpSz; ++i)
        detA *= U[i][i];
    output << pow(-1.0, 1.0 * cntSwitchRowsAndColumns) * detA << endl << endl;              
    TMatrix reverseA(_matrA.GetSizeRow(), _matrA.GetSizeCol(), Zero);
    for (int i = 0; i < tmpSz; ++i) {
        TVector vec(_matrA.GetSizeRow());
        TVector calcVec;        
        vec[i] = 1.0;        
        calcVec.SetLink(_solveAx_is_b(L, U, vec));		        
        reverseA.AssignColumn(calcVec, i);
        calcVec.Clear();
        vec.Clear();
    }
    _writeToFile(pathTo); 
    reverseA.Print(output, "A^(-1)");          
    TMatrix check(_matrA * reverseA);
    check.Print(log, "A * A^(-1)");    
    L.Clear();
    U.Clear();
    reverseA.Clear();
    check.Clear();
    _clear();
    delete tmpRead; 
    return 0;
}

TVector TSolve::_solveAx_is_b(const TMatrix& mL, 
                              const TMatrix& mU, 
                              const TVector& vB) {
    int tmpSz = vB.GetSize();    
	TVector z(tmpSz);
    TVector x(tmpSz);
	z[0] = vB[0];	
    for (int i = 1; i < tmpSz; ++i) {
        double tmpVal = 0.0;
        for (int j = 0; j < i; ++j)
            tmpVal += mL[i][j] * z[j];
        z[i] = vB[i] - tmpVal;
    }    
    x[tmpSz - 1] = z[tmpSz - 1] / mU[tmpSz - 1][tmpSz - 1];    
    for (int i = tmpSz - 1; i >= 0; --i) {
        double tmpVal = 0.0;
        for (int j = i + 1; j < tmpSz; ++j) {
            tmpVal += mU[i][j] * x[j];
        }
        x[i] = (z[i] - tmpVal) / mU[i][i];
    }
    z.Clear();
    return x;
}
\end{lstlisting}
\vspace{0.5cm}

\textbf{Тестирование}\\

\textbf{Входной файл}
\begin{verbatim}
4
9 -5 -6 3 -8
1 -7 1 0 38
3 -4 9 0 47
6 -1 9 8 -8
\end{verbatim}

\textbf{Выходной файл}
\begin{verbatim}
solve1Gauss.log:

|Method Gauss (LUP)| by Alexander Bales 80-308

Matrix L:
1 0 0 0 
0.111111 1 0 0 
0.666667 -0.362069 1 0 
0.333333 0.362069 0.764259 1 

Matrix U:
9 -5 -6 3 
0 -6.44444 1.66667 -0.333333 
0 0 13.6034 5.87931 
0 0 0 -5.37262 

Matrix A * A^(-1):
1 5.55112e-17 0 2.22045e-16 
6.93889e-18 1 -1.73472e-18 6.93889e-17 
1.11022e-16 1.11022e-16 1 4.44089e-16 
0 1.38778e-16 -2.77556e-17 1 

res:

Matrix A:
9 -5 -6 3 
1 -7 1 0 
6 -1 9 8 
3 -4 9 0 

Vector B = (-8, 38, -8, 47)

Vector X = (0, -5, 3, -5)

Matrix A^(-1):
0.111 -0.149 -0.0418 0.133 
0.0113 -0.168 -0.00425 0.0304 
-0.0321 -0.0248 0.012 0.0804 
-0.046 0.119 0.142 -0.186 
\end{verbatim}

\pagebreak
