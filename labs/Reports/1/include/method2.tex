\textbf{Метод 2 --- Метод прогонки}\\

\textbf{Задание}

Реализовать метод прогонки в виде программы, задавая в качестве входных данных ненулевые элементы матрицы системы и вектор правых частей. Используя разработанное программное обеспечение, решить СЛАУ с трехдиагональной матрицей.\\

\textbf{Вариант:} 3

$
\begin{cases}
13x_1-5x_2=-66\\
4x_1+9x_2-5x_3=-47\\
-x_2-12x_3-6x_4=-43\\
6x_3+20x_4-5x_5=-74\\
4x_4+5x_5=14
\end{cases}
$
\vspace{0.5cm}

\textbf{Описание алгоритма}

Метод прогонки является одним из эффективных методов решения СЛАУ с трех--диагональными матрицами, возникающих при конечно--разностной аппроксимации задач для обыкновенных дифференциальных уравнений (ОДУ) и уравнений в частных производных второго порядка и является частным случаем метода Гаусса. Рассмотрим следующую СЛАУ:\\

$
\begin{cases}
a_1=0\\
b_1x_1+c_1x_2=d_1\\
a_2x_1+b_2x_2+c_2x_3=d_2\\
a_3x_2+b_3x_3+c_3x_4=d_3\\
...\\
a_{n-1}x_{n-2}+b_{n-1}x_{n-1}+c_{n-1}x_n=d_{n-1}\\
a_nx_{n-1}+b_nx_n=d_n\\
c_n=0
\end{cases}
$\\

решение которой будем искать в виде:

\begin{equation}
x_i=P_ix_{i+1}+Q_i, i=\overline{1,n}
\end{equation}

где $P_i, Q_i, i=\overline{1,n}$ --- прогоночные коэффициенты, подлежащие определению.

$$
P_i=\frac{-c_i}{b_i+a_iP_{i-1}}, Q_i=\frac{d_i-a_iQ_{i-1}}{b_i+a_iP_{i-1}}, i=\overline{2,n-1}
$$

$$
P_1=\frac{-c_1}{b_1}, Q_1=\frac{d_1}{b_1}, i=1
$$

$$
P_n=0, Q_n=\frac{d_n-a_nQ_{n-1}}{b_n+a_nP_{n-1}}, i=n
$$

Обратный ход метода прогонки осуществляется в соответствии с выражением (1):

$$
\begin{cases}
x_n=P_nx_{n+1}+Q_n=0 \cdot x_{n+1}+Q_n=Q_n\\
x_{n-1}=P_{n-1}x_n+Q_{n-1}\\
...\\
x_1=P_1x_2+Q_1
\end{cases}
$$\\

Общее число операций в методе прогонки равно $8n+1$, т.е. пропорционально числу уравнений. Такие методы решения СЛАУ называют \textit{экономичными}. Для сравнения число операций в методе Гаусса пропорционально $n^3$.\\

Для устойчивости метода прогонки достаточно выполнение следующих условий:

$$
a_i \neq 0, c_i \neq 0, i=\overline{2,n-1}
$$

$$
|b_i| \geq |a_i|+|c_i|, i=\overline{1,n}
$$\\

\textbf{Реализация}

\begin{lstlisting}
int TSolve::ToSolveByTripleDiagMatrix() {
    TFRFF* tmpRead = _readFromFile(pathFrom, TripleDiagMatrix);
    if (tmpRead == NULL)
        return -1;
    _matrA.SetLink(tmpRead->matr);
    _vecB.SetLink(tmpRead->vec);
    ofstream log("solve1TripleDiagMatrix.log", ios::out);
    log << "|Method TripleDiagMatrix| by Alexander Bales 80-308" << endl << endl;
    double P, Q;
    try {
        P = -_matrA[0][1] / _matrA[0][0];
        Q = _vecB[0] / _matrA[0][0];            
    }
    catch (const out_of_range& e) {
        cerr << "Out of range: " << e.what() << endl;
    }
    _vecX = _vecB;    
    _findSolve(P, Q, 1, _vecX, log);        
    _writeToFile(pathTo);
    _clear();
    delete tmpRead; 
    return 0;
}

void TSolve::_findSolve(double P, double Q, int n, TVector& x, ofstream& log) {
    if (n == x.GetSize()) {
        x[n - 1] = Q;
        log << "P_" << n << " = " << 0 << endl;
        log << "Q_" << n << " = " << Q << endl;
    } else {
        try {
            _findSolve(-_matrA[n][2] / (_matrA[n][0] * P + _matrA[n][1]),
                        (_vecB[n] - _matrA[n][0] * Q) / (_matrA[n][0] * P + _matrA[n][1]),
                        n + 1, x, log);
            log << "P_" << n << " = " << P << endl;
            log << "Q_" << n << " = " << Q << endl;
            x[n - 1] = P * x[n] + Q;           
        }
        catch (const out_of_range& e) {
            cerr << e.what() << endl;
        }
    }        
}
\end{lstlisting}
\vspace{0.5cm}

\textbf{Тестирование}\\

\textbf{Входной файл}
\begin{verbatim}
5
13 -5 0 -66
-4 9 5 -47
-1 -12 -6 -43
6 20 -5 -74
0 4 5 14
\end{verbatim}

\textbf{Выходной файл}
\begin{verbatim}
solve1TripleDiagMatrix.log:

|Method TripleDiagMatrix| by Alexander Bales 80-308

P_5 = 0
Q_5 = 3.5
P_4 = 0.29722
Q_4 = -6.03646
P_3 = -0.529572
Q_3 = 4.59145
P_2 = -0.670103
Q_2 = -9.02062
P_1 = 0.384615
Q_1 = -5.07692

res:

Matrix A:
13 -5 0 
-4 9 5 
-1 -12 -6 
6 20 -5 
0 4 5 

Vector B = (-66, -47, -43, -74, 14)

Vector X = (-10.4, -13.9, 7.24, -5, 3.5)
\end{verbatim}

\pagebreak
