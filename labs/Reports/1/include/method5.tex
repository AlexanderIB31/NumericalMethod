\textbf{Метод 4 --- Метод вращений}\\

\textbf{Задание}

Реализовать метод вращений в виде программы, задавая в качестве входных данных матрицу и точность вычислений. Используя разработанное программное обеспечение, найти собственные значения и собственные векторы симметрических матриц. Проанализировать зависимость погрешности вычислений от числа итераций.\\

\textbf{Вариант:} 3

$
\begin{pmatrix}
5 & 5 & 3\\
5 & -4 & 1\\
3 & 1 & 2
\end{pmatrix}
$
\vspace{0.5cm}

\textbf{Описание алгоритма}

Метод вращений Якоби применим только для симметрических матриц $A_{n \times n}$ ($A=A^T)$ и решает полную проблему собственных значений и собственных векторов таких матриц. Он основан на отыскании с помощью итерационных процедур матрицы $U$ в преобразовании подобия $\Lambda = U^{-1}AU$, а поскольку для симметрических матриц $A$ матрица преобразования подобия $U$ является ортогональной ($U^{-1}=U^T$), то $\Lambda = U^TAU$, где $\Lambda$ --- диагональная матрица с собственными значениями на главной диагонали.

$$
\Lambda = \begin{pmatrix}
\lambda_1 & \cdots & 0\\
\cdots & \ddots & \cdots\\
0 & \cdots & \lambda_n
\end{pmatrix}
$$\\

Пусть дана симметрическая матрица $A$. Требуется для нее вычислить с точностью $\varepsilon$ все собственные значения и соответствующие им собственные векторы. Алгоритм метода вращений следующий:\\

Пусть известна матрица $A^{(k)}$ на $k$--й итерации, при этом для $k=0$ $A^{(0)}=A$.\\
1. Выбирается максимальный по модулю недиагональный элемент $a_{ij}^{(k)}$ матрицы $A^{(k)} (|a_{ij}^{(k)}|)=\max\limits_{l<m}|a_{lm}^{(k)}|$.\\
2. Ставится задача найти такую ортогональную матрицу $U^{(k)}$, чтобы в результате преобразования подобия $A{(k+1)}=U^{(k)T}A^{(k)}U^{(k)}$ произошло обнуление элемента $a_{ij}^{(k+1)}$ матрицы $A^{(k+1)}$. В качестве ортогональной матрицы выбирается матрица вращения, имеющая следующий вид:

$$
U^{(k)} = \bordermatrix{
~ & i & j \cr
i & cos \varphi^{(k)} & -sin \varphi^{(k)} \cr
j & sin \varphi^{(k)} & cos \varphi^{(k)}
}
$$

Угол вращения $\varphi^{(k)}$ определяется из условия $a_{ij}^{(k+1)}=0$:

$$
\varphi^{(k)}=\frac{1}{2}arctg\frac{2a_{ij}^{(k)}}{a_{ii}^{(k)}-a_{jj}^{(k)}}; a_{ii}^{(k)}=a_{jj}^{(k)}, \varphi^{(k)}=\frac{\pi}{4}
$$

3. Строится матрица $A^{(k+1)}$:

$$
A^{(k+1)}=U^{(k)T}A^{(k)}U^{(k)}
$$

в которой элемент $a_{ij}^{(k+1)} \approx 0$.\\

В качестве критерия окончания итерационного процесса используется условие малости суммы квадратов внедиагональных элементов:

$$
t(A^{(k+1)})=(\sum_{l,m;l<m}(a_{lm}^{(k+1)})^2)^\frac{1}{2}
$$\\

\textbf{Реализация}

\begin{lstlisting}
int TSolve::ToSolveByRotateMethod() {
    TFRFF* tmpRead = _readFromFile(pathFrom, Rotate);
    if (tmpRead == NULL)
        return -1;
    _matrA.SetLink(tmpRead->matr);
    ofstream log("solve1RotateMethod.log", ios::out);
    log << "|Method Rotate| by Alexander Bales 80-308" << endl << endl;
    TMatrix rotateMatr( _matrA.GetSizeRow(), 
                        _matrA.GetSizeCol(),
                        Identity );
    TMatrix A(_matrA);
    TMatrix OwnVectors(rotateMatr);
    //ofstream urs("tmp.log", ios::out);
    int cnt = 0;
    while (_t(A) > eps) {
        cout << _t(A) << endl;
        log << "|" << cnt++ + 1 << " iteration|" << endl;
        log << "********************************************" << endl;
        A.Print(log, "A");
        pair<int, int> pos = A.FindPosMaxNotDiagElem();
        int i = pos.first, j = pos.second;
        log << "maxPos = (" << i + 1 << "; " << j + 1 << ");" << endl << endl;
        double angel = 0.0;
        try {
            angel = A[i][i] == A[j][j] ? M_PI / 4 :
                            .5 * atan(2 * A[i][j] /
                                    (A[i][i] - A[j][j]));
        }
        catch (const out_of_range& e) {
            cerr << "Out of range: " << e.what() << endl;
        }
        rotateMatr[i][i] = rotateMatr[j][j] = cos(angel);
        rotateMatr[i][j] = -sin(angel);
        rotateMatr[j][i] = -rotateMatr[i][j];
        rotateMatr.Print(log, "rotateMatr");
        A = rotateMatr.Rotate() * A * rotateMatr;
        OwnVectors = OwnVectors * rotateMatr;
        rotateMatr[i][i] = rotateMatr[j][j] = 1.0;
        rotateMatr[i][j] = rotateMatr[j][i] = 0.0;
        OwnVectors.Print(log, "curMultiplyRotateMatrix");
        log << "############################################" << endl << endl;
    }
    _writeToFile(pathTo);
    for (int i = 0; i < min(A.GetSizeRow(), A.GetSizeCol()); i++)
        output << "l_" << i + 1 << " = " << A[i][i] << endl;
    output << endl;
    for (int i = 0; i < OwnVectors.GetSizeCol(); i++) {
        output << "x_" << i << " = (";
        for (int j = 0; j < OwnVectors.GetSizeRow(); j++) {
            output << OwnVectors[j][i];
            if (j != OwnVectors.GetSizeRow() - 1)
                output << ", ";
        }
        output << ");" << endl;
    }
    _clear();
    A.Clear();
    OwnVectors.Clear();
    rotateMatr.Clear(); 
    return 0;
}
\end{lstlisting}
\vspace{0.5cm}

\textbf{Тестирование}\\

\textbf{Входной файл}
\begin{verbatim}
3
5 5 3
5 -4 1
3 1 2
0.001
\end{verbatim}

\textbf{Выходной файл}
\begin{verbatim}
solve1RotateMethod.log:
|Method Rotate| by Alexander Bales 80-308

|1 iteration|
********************************************
Matrix A:
5 5 3 
5 -4 1 
3 1 2 

maxPos = (1; 2);

Matrix rotateMatr:
0.9135 -0.406839 0 
0.406839 0.9135 0 
0 0 1 

Matrix curMultiplyRotateMatrix:
0.9135 -0.406839 0 
0.406839 0.9135 0 
0 0 1 

############################################

|2 iteration|
********************************************
Matrix A:
7.22681 4.44089e-16 3.14734 
4.44089e-16 -6.22681 -0.307016 
3.14734 -0.307016 2 

maxPos = (1; 3);

Matrix rotateMatr:
0.905216 0 -0.424952 
0 1 0 
0.424952 0 0.905216 

Matrix curMultiplyRotateMatrix:
0.826915 -0.406839 -0.388194 
0.368277 0.9135 -0.172887 
0.424952 0 0.905216 

############################################

|3 iteration|
********************************************
Matrix A:
8.70433 -0.130467 0 
-0.130467 -6.22681 -0.277915 
8.32667e-17 -0.277915 0.522485 

maxPos = (2; 3);

Matrix rotateMatr:
1 0 0 
0 0.999156 -0.0410727 
0 0.0410727 0.999156 

Matrix curMultiplyRotateMatrix:
0.826915 -0.422439 -0.371157 
0.368277 0.905628 -0.210261 
0.424952 0.0371796 0.904452 

############################################

|4 iteration|
********************************************
Matrix A:
8.70433 -0.130357 0.00535863 
-0.130357 -6.23824 0 
0.00535863 -2.08167e-17 0.53391 

maxPos = (1; 2);

Matrix rotateMatr:
0.999962 0.00872288 0 
-0.00872288 0.999962 0 
0 0 1 

Matrix curMultiplyRotateMatrix:
0.830568 -0.41521 -0.371157 
0.360363 0.908806 -0.210261 
0.424612 0.040885 0.904452 

############################################

|5 iteration|
********************************************
Matrix A:
8.70546 -1.38778e-17 0.00535843 
0 -6.23937 4.67427e-05 
0.00535843 4.67427e-05 0.53391 

maxPos = (3; 1);

Matrix rotateMatr:
1 0 -0.000655741 
0 1 0 
0.000655741 0 1 

Matrix curMultiplyRotateMatrix:
0.830324 -0.41521 -0.371701 
0.360225 0.908806 -0.210497 
0.425205 0.040885 0.904173 

############################################

res:
Matrix A:
5 5 3 
5 -4 1 
3 1 2 

l_1 = 8.71
l_2 = -6.24
l_3 = 0.534

x_0 = (0.83, 0.36, 0.425);
x_1 = (-0.415, 0.909, 0.0409);
x_2 = (-0.372, -0.21, 0.904);

\end{verbatim}

\pagebreak
