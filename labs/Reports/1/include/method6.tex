\textbf{Метод 5 --- QR--разложение}\\

\textbf{Задание}

Реализовать алгоритм QR--разложения матриц в виде программы. На его основе разработать программу, реализующую QR--алгоритм решения полной проблемы собственных значений произвольных матриц, задавая в качестве входных данных матрицу и точность вычислений. С использованием разработанного программного обеспечения найти собственные значения матрицы.\\

\textbf{Вариант:} 3

$
\begin{pmatrix}
5 & -5 & -6\\
-1 & -8 & -5\\
2 & 7 & -3
\end{pmatrix}
$
\vspace{0.5cm}

\textbf{Описание алгоритма}

При решении полной проблемы собственных значений для несимметричных матриц эффективным является подход, основанный на приведении матриц к подобным, имеющим треугольный или квазитреугольный вид. Одним из наиболее распространенных методов этого класса является QR--алгоритм, позволяющий находить как вещественные, так и комплексные собственные значения.\\

В основе QR--алгоритма лежит представление матрицы в виде $A=QR$, где $Q$ --- ортогональная матрица ($Q^{-1}=Q^T$), а $R$ --- верхняя треугольная. Такое разложение существует для любой квадратной матрицы. Одним из возможных подходов к построению QR разложения является использование преобразования Хаусхолдера, позволяющего обратить в нуль группу поддиагональных элементов столбца матрицы.\\

Преобразование Хаусхолдера осуществляется с использованием матрицы Хаусхолдера, имеющей следующий вид:

$$
H=E-\frac{2}{v^Tv}vv^T
$$

где $v$ --- произвольный ненулевой вектор--столбец, $E$ --- единичная матрица, $vv^T$ --- квадратная матрица того же размера.\\

Вектор $v$ определяется следующим образом:

$$
v=b+sign(b_1)\|b\|_2e_1
$$

где $\|b\|_2=(\sum\limits_ib_i^2)^\frac{1}{2}$ --- евклидова норма вектора, $e_1=(1,0,...,0)^T$.\\

Применяя описанную процедуру с целью обнуления поддиагональных элементов каждого из столбцов исходной матрицы, можно за фиксированное число шагов получить ее QR--разложение.\\

Процедура QR--разложения многократно используется в QR--алгоритме вычисления собственных значений. Строится следующий итерационный процесс:\\

$A^{(0)}=A$\\
$A^{(0)}=Q^{(0)}R^{(0)}$ --- производится QR--разложение\\
$A^{(1)}=R^{(0)}Q^{(0)}$ --- производится перемножение матриц\\
...\\
$A^{(k)}=Q^{(k)}R^{(k)}$ --- разложение\\
$A^{(k+1)}=R^{(k)}Q^{(k)}$ --- перемножение\\

Таким образом, каждому вещественному собственному значению будет соответствовать столбец со стремящимися к нулю поддиагональными элементами и в качестве критерия сходимости итерационного процесса для таких собственных значений можно использовать следующее неравенство:

$$
(\sum_{l=m+1}(a_{lm}^{(k)})^2)^\frac{1}{2} \leq \varepsilon
$$

При этом соответствующее собственное значение принимается равным диагональному элементу данного столбца.\\

Если в ходе итераций прослеживается комплексно--сопряженная пара собственных значений, соответствующая блоку, образуемому элементами $j$--го и $(j+1)$--го столбцов $a_{jj}^{(k)}, a_{jj+1}^{(k)}, a_{j+1j}^{(k)}, a_{j+1j+1}^{(k)}$, то несмотря на значительное изменение в ходе итераций самих этих элементов, собственные значения, соответствующие данному блоку и определяемые из решения квадратного уравнения $(a_{jj}^{(k)}-\lambda^{(k)})(a_{j+1j+1}^{(k)}-\lambda^{(k)})=a_{jj+1}^{(k)}a_{j+1j}^{(k)}$, начиная c некоторого $k$, отличаются незначительно. В качестве критерия окончания итераций для таких блоков может быть использовано следующее условие $|\lambda^{(k)}-\lambda^{(k-1)}| \leq \varepsilon$.\\

\textbf{Реализация}

\begin{lstlisting}

\end{lstlisting}
\vspace{0.5cm}

\textbf{Тестирование}\\

\textbf{Входной файл}
\begin{verbatim}
3
5 -5 -6
-1 -8 -5
2 7 -3
0.001
\end{verbatim}

\textbf{Выходной файл}
\begin{verbatim}
solve1QR.log:
|Method QR| by Alexander Bales 80-308

3.03333 5.19283 5.11535 
0.71339 -5.70348 10.9534 
1.73876 -2.92221 -3.32986 

5.79843 2.11649 -1.48932 
0.575435 -4.01297 -8.26022 
5.19363 5.04254 -7.78546 

1.55173 8.27555 7.6834 
-3.83406 -2.73465 5.423 
4.96303 -2.94102 -4.81708 

-3.14749 7.58946 -1.34069 
0.526922 -5.21176 -10.7171 
-3.99439 2.2088 2.35925 

-2.20408 8.58656 -0.389965 
-4.89394 -8.16322 -0.455668 
-6.52049 -2.6219 4.3673 

-2.49884 4.71682 -8.49962 
-6.74093 -5.65205 -7.03694 
-2.36348 0.383827 2.15088 

-8.11358 7.18704 4.65476 
-4.5264 -3.58066 -5.25446 
-1.9268 1.33936 5.69424 

-5.33289 8.49936 -4.29515 
-4.61994 -5.4185 6.67382 
-0.9889 0.274396 4.75139 

-3.2153 6.87261 -2.15913 
-5.16699 -7.19904 -8.79677 
-0.609817 -0.0419114 4.41434 

-6.11476 4.05064 7.46761 
-8.4366 -5.04646 2.2089 
-0.512829 0.115725 5.16122 

-7.17506 7.22887 -7.01184 
-5.29099 -3.51479 4.76206 
-0.231564 0.14895 4.68985 

-5.02926 8.31725 2.50335 
-4.01722 -5.78925 -8.02337 
-0.124846 0.0359831 4.81852 

-3.28602 6.2672 2.90602 
-6.2133 -7.6045 7.68895 
-0.0937216 -0.00772327 4.89051 

-6.52824 4.40073 -8.3292 
-8.02614 -4.25755 -1.08913 
-0.0627453 0.0237435 4.78578 

-6.97305 7.72097 5.99533 
-4.69775 -3.87494 -5.76463 
-0.0294156 0.021379 4.84799 

-4.59877 8.21127 -1.79164 
-4.23437 -6.23585 8.1257 
-0.016857 0.00435829 4.83461 

-3.35687 5.57105 -4.20784 
-6.85538 -7.4661 -7.21033 
-0.0129455 -0.000149771 4.82297 

-7.19273 4.99117 8.30146 
-7.44142 -3.64487 -0.613455 
-0.00811517 0.00422724 4.8376 

-6.57876 8.02773 -5.34196 
-4.40678 -4.25095 6.39708 
-0.00379228 0.00279957 4.8297 

-4.18394 7.98265 0.872999 
-4.44798 -6.64698 -8.28887 
-0.00229859 0.0004847 4.83091 

-3.72875 4.87821 5.4361 
-7.55493 -7.10406 6.3125 
-0.00181818 0.000117299 4.83281 

-7.5035 5.68994 -8.06995 
-6.74255 -3.32729 2.08056 
-0.00104359 0.000658023 4.83078 

-6.16162 8.23673 4.61023 
-4.19537 -4.67014 -6.94114 
-0.000503125 0.000339114 4.83175 

-6.16162 8.23673 4.61023 
-4.19537 -4.67014 -6.94114 
-0.000503125 0.000339114 4.83175 

res:
Matrix A:
5 -5 -6 
-1 -8 -5 
2 7 -3 

eigenvalue[1] = -5.42 + 5.83i
eigenvalue[2] = -5.42 - 5.83i
eigenvalue[3] = 4.83
\end{verbatim}

\pagebreak
