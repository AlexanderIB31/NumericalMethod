\textbf{Метод 1 --- Интерполяция}\\

\textbf{Задание}

Используя таблицу значений $Y_i$ функции $y=f(x)$, вычисленных в точках $X_i$, $i=0,...,3$ построить интерполяционные многочлены Лагранжа и Ньютона, проходящие через точки ${X_i, Y_i}$. Вычислить значение погрешности интерполяции в точке $X^*$.\\

\textbf{Вариант:} 3

$y=tg(x)$, а) $X_i=0, \frac{\pi}{8}, \frac{2\pi}{8}, \frac{3\pi}{8}$; б) $X_i=0, \frac{\pi}{8}, \frac{\pi}{3}, \frac{3\pi}{8}$; $X^*=\frac{3\pi}{16}$\\

\textbf{Описание алгоритма}

Пусть на отрезке $[a, b]$ задано множество несовпадающих точек $x_i$ (интерполяционных узлов), в которых известны значения функции $f_i=f(x_i), i=0,...,n$. Приближающая функция $\varphi(x, a)$ такая, что выполняются равенства

$$
\varphi(x_i, a_0, ..., a_n)=f(x_i)=f_i, i=0,...,n
$$

называется интерполяционной.

Наиболее часто в качестве приближающей функции используют многочлены степени $n$:

$$
P_n(x)=\sum\limits_{i=0}^na_ix^i
$$

Произвольный многочлен может быть записан в виде:

$$
L_n(x)=\sum\limits_{i=0}^nf_il_i(x)
$$

Здесь $l_i(x)$ --- многочлены степени $n$, так называемые лагранжевы многочлены влияния, которые удовлетворяют условию $l_i(x_j)=\begin{cases}
1, i = j\\
0, i \neq j
\end{cases}$ и, соответственно, $l_i(x)=\prod\limits_{j=0,j \neq i}^n\frac{(x-x_i)}{(x_i-x_j)}$, а интерполяционный многочлен запишется в виде:

$$
L_n(x)=\sum\limits_{i=0}^nf_i\prod\limits_{j=0, j \neq i}^n\frac{(x-x_i)}{(x_i-x_j)}
$$

\pagebreak

Данный интерполяционный многочлен называется интерполяционным многочленом Лагранжа.\\

Введем понятие разделенной разности. Разделенные разности нулевого порядка совпадают со значениями функции в узлах. Разделенные разности первого порядка обозначаются $f(x_i,x_j)$ и определяются через разделенные разности нулевого порядка:

$$
f(x_i, x_j)=\frac{f_i-f_j}{x_i-x_j}
$$

разделенные разности второго порядка определяются через разделенные разности первого порядка:

$$
f(x_i,x_j,x_k)=\frac{f(x_i,x_j)-f(x_j,x_k)}{x_i-x_k}
$$

Разделенная разность порядка $n-k+2$ определяется соотношениями:

$$
f(x_i,x_j,x_k,...,x_{n-1},x_n)=\frac{f(x_i,x_j,x_k,...,x_{n-1})-f(x_j,x_k,...,x_n)}{x_i-x_n}
$$

Пусть известны значения аппроксимируемой функции $f(x)$ в точках $x_0,x_1,...,x_n$. Интерполяционный многочлен, значения которого в узлах интерполяции совпадают со значениями функции $f(x)$ может быть записан в виде:\\

$P_n(x)=f(x_0)+(x-x_0)f(x_1,x_0)+(x-x_0)(x-x_1)f(x_0,x_1,x_2)+...+$\\
$+(x-x_0)(x-x_1)...(x-x_n)f(x_0,x_1,...,x_n)$\\

Данная запись многочлена есть так называемый интерполяционный многочлен Ньютона.\\

\textbf{Реализация}\\

\textbf{Лагранж}
\begin{lstlisting}
void PolynomialLagrange() {
	auto Y = [](double x) -> double { return tan(x); };
	auto Product = [](const vector<double>& X, double curX, size_t pos) -> double {
		double temp = 1.0;
		for (size_t i = 0; i < 4; ++i) {
			if (i == pos) continue;
			temp *= (curX - X[i]) / (X[pos] - X[i]);
		}
		return temp;
	};

	vector<double> X;
	// task a)
	/*
	X.push_back(0);
	X.push_back(1.0 * M_PI / 8);
	X.push_back(2.0 * M_PI / 8);
	X.push_back(3.0 * M_PI / 8);
	*/
	//task b)
	X.push_back(0);
	X.push_back(1.0 * M_PI / 8);
	X.push_back(1.0 * M_PI / 3);
	X.push_back(3.0 * M_PI / 8);

	double perfectX = 3.0 * M_PI / 16;
	double sum = 0.0;
	for (size_t i = 0; i < 4; ++i) {
		sum += Y(X[i]) * Product(X, perfectX, i);	
	}
	cout << "L(" << perfectX << ") = " << sum << endl;
	cout << "y(" << perfectX << ") = " << Y(perfectX) << endl;
	cout << "|L(" << perfectX << ") - y(" << perfectX << ")| = " << fabs(Y(perfectX) - sum) << endl;
}

\end{lstlisting}
\vspace{0.5cm}

\textbf{Ньютон}
\begin{lstlisting}
void PolynomialNewtoon() {
	auto Y = [](double x) -> double { return sin(M_PI * x / 6); };
	function<double(const vector<double>&, double, size_t, size_t)> FuncY; 
	FuncY = [Y, &FuncY](const vector<double>& X, double curX, size_t start, size_t end) -> double {
		if (start == end) {
			return Y(X[start]);
		}
		else {
			return (FuncY(X, curX, start, end - 1) - FuncY(X, curX, start + 1, end)) / (X[start] - X[end]);
		}
	};
	auto Product = [](const vector<double>& X, double curX, size_t sz) -> double {
		double prod = 1.0;
		for (size_t i = 0; i < sz; ++i)
			prod *= (curX - X[i]);
		return prod;
	};

	vector<double> X;
	// task a)
	/*
	X.push_back(0);
	X.push_back(1.0 * M_PI / 8);
	X.push_back(2.0 * M_PI / 8);
	X.push_back(3.0 * M_PI / 8);
	*/
	//task b)
	X.push_back(0);
	X.push_back(1.0 * M_PI / 8);
	X.push_back(1.0 * M_PI / 3);
	X.push_back(3.0 * M_PI / 8);
	

	double perfectX = 1.5;
	double sum = 0.0;
	for (size_t i = 1; i < 4; ++i) {
		sum += FuncY(X, perfectX, 0, i) * Product(X, perfectX, i);
	}
	cout << "P(" << perfectX << ") = " << sum << endl;
	cout << "y(" << perfectX << ") = " << Y(perfectX) << endl;
	cout << "|P(" << perfectX << ") - y(" << perfectX << ")| = " << fabs(Y(perfectX) - sum) << endl;

}
\end{lstlisting}
\vspace{0.5cm}

\textbf{Тестирование}\\


\textbf{Выходной файл}
\begin{verbatim}
а)
Polynomial Lagrange
L(0.589049) = 0.644607
y(0.589049) = 0.668179
|L(0.589049) - y(0.589049)| = 0.0235719
****************************************
Polynomial Newtoon
P(1.5) = 0.70664
y(1.5) = 0.707107
|P(1.5) - y(1.5)| = 0.000467104
****************************************

б)
Polynomial Lagrange
L(0.589049) = 0.585251
y(0.589049) = 0.668179
|L(0.589049) - y(0.589049)| = 0.0829278
****************************************
Polynomial Newtoon
P(1.5) = 0.706792
y(1.5) = 0.707107
|P(1.5) - y(1.5)| = 0.000314796
****************************************
\end{verbatim}


\pagebreak
