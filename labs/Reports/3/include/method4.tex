\textbf{Метод 4 --- Численное дифференцирование}\\

\textbf{Задание}

Вычислить первую и вторую производную от таблично заданной функции $y_i=f(x_i), i=0,1,2,3,4$ в точке $x=X^*$.\\

\textbf{Вариант:} 3

$X^*=2.0$\\
\begin{tabular}{|c|c|c|c|c|c|}
\hline
i & 0 & 1 & 2 & 3 & 4 \\
\hline
$x_i$ & 1.0 & 1.5 & 2.0 & 2.5 & 3.0 \\
\hline
$y_i$ & 0.0 & 0.40547 & 0.69315 & 0.91629 & 1.0986 \\
\hline
\end{tabular}
\vspace{0.5cm}

\textbf{Описание алгоритма}

Формулы численного дифференцирования в основном используется при нахождении производных от функции $y=f(x)$, заданной таблично. Исходная функция $y_i=f(x_i), i=0,1,...,M$ на отрезках $[x_j, x_{j+k}]$ заменяется некоторой приближающей, легко вычисляемой функцией $\varphi(x,\overline{a}), y=\varphi(x,\overline{a})+R(x)$, где $R(x)$ --- остаточный член приближения, $\overline{a}$ --- набор коэффициентов. Наиболее часто в качестве приближающей функции $\varphi(x,\overline{a})$ берется интерполяционный многочлен $\varphi(x,\overline{a})=P_n(x)=\sum\limits_{i=0}^na_ix^i$, а производные соответствующих порядков определяются дифференцированием многочлена.

Первая производная:

$$
y'(x) \approx \varphi'(x)=\frac{y_{i+1}-y_i}{x_{i+1}-x_i}+\frac{\frac{y_{i+2}-y_{i+1}}{x_{i+2}-x_{i+1}}-\frac{y_{i+1}-y_i}{x_{i+1}-x_i}}{x_{i+2}-x_i}(2x-x_i-x_{i+1}), x \in [x_i, x_{i+1}]
$$

Вторая производная:

$$
y''(x) \approx \varphi''(x)=2\frac{\frac{y_{i+2}-y_{i+1}}{x_{i+2}-x_{i+1}}-\frac{y_{i+1}-y_i}{x_{i+1}-x_i}}{x_{i+2}-x_i}, x \in [x_i, x_{i+1}]
$$\\

\textbf{Реализация}
\begin{lstlisting}
#include <iostream>
#include <functional>
#include <fstream>
#include <vector>

using namespace std;

int main(int argc, char* argv[]) {
	auto firstDerivative2 = [](const vector<double>& Y, const vector<double>& X, int pos, double x) -> double {
		if (pos >= Y.size() - 2) return 0;
		return (Y[pos + 1] - Y[pos]) / (X[pos + 1] - X[pos]) + 
			((Y[pos + 2] - Y[pos + 1]) / (X[pos + 2] - X[pos + 1]) - (Y[pos + 1] - Y[pos]) / (X[pos + 1] - X[pos])) 
				/ (X[pos + 2] - X[pos]) * (2 * x - X[pos] - X[pos + 1]);
	};
	auto secondDerivative2 = [](const vector<double>& Y, const vector<double>& X, int pos, double x) -> double {
		if (pos >= Y.size() - 2) return 0;
		return 2 * ((Y[pos + 2] - Y[pos + 1]) / (X[pos + 2] - X[pos + 1]) 
				- (Y[pos + 1] - Y[pos]) / (X[pos + 1] - X[pos])) 
			/ (X[pos + 2] - X[pos]);
	};

	if (argc != 2) {
		cerr << "Error: arg is incorrect" << endl;
		exit(-1);
	}

	string dataFile = argv[1];
	vector<double> X, Y;
	size_t N;
	double temp, perfectX;
	ifstream in(dataFile, ios::in);

	in >> N;
	for (size_t i = 0; i < N; ++i) {
		in >> temp;
		X.push_back(temp);
	}
	for (size_t i = 0; i < N; ++i) {
		in >> temp;
		Y.push_back(temp);
	}
	in >> perfectX;
	in.close();

	for (size_t i = 0; i < N - 1; ++i) {
		if (X[i] <= perfectX && perfectX <= X[i + 1]) {
			cout << "F'(x) = " << firstDerivative2(Y, X, i, perfectX) << endl;
			cout << "F\"(x) = " << secondDerivative2(Y, X, i, perfectX) << endl;
			break;
		}
	}
	return 0;
}
\end{lstlisting}
\vspace{0.5cm}

\textbf{Тестирование}\\

\textbf{Входной файл}
\begin{verbatim}
5
1.0 1.5 2.0 2.5 3.0
0.0 0.40547 0.69315 0.91629 1.0986
2.0
\end{verbatim}

\textbf{Выходной файл}
\begin{verbatim}
F'(x) = 0.51082
F"(x) = -0.25816
\end{verbatim}

\pagebreak
