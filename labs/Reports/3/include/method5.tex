\textbf{Метод 5 --- Численное интегрирование}\\

\textbf{Задание}

Вычислить определенный интеграл $F=\int\limits_{X_0}^{X_1}ydx$, методами прямоугольников, трапеций, Симпсона с шашами $h_1, h_2$. Оценить погрешность вычислений, используя метод Рунге-Ромберга.\\

\textbf{Вариант:} 3

$y=\frac{x}{(3x+4)^3}, \quad X_0=-1, X_k=1, h_1=0.5, h_2=0.25$\\

\textbf{Описание алгоритма}

Формулы численного интегрирования используются в тех случаях, когда вычислить аналитически интеграл $F=\int\limits_{a}^bf(x)dx$ не удается. Отрезок $[a, b]$ разбивают точками $x_0,...,x_N$ так, что $a=x_0 \leq x_1 \leq ... \leq x_N=b$ с достаточно мелким шагом $h_i=x_i-x_{i-1}$ и на одном или нескольких отрезках $h_i$ подынтегральную функцию $f(x)$ заменяют такой приближающей $\varphi(x)$ так, что она, во--первых, близка $f(x)$, а, во--вторых, интеграл от $\varphi(x)$ легко вычисляется.

Заменим подынтегральную функцию интерполяционным многочленом Лагранжа нулевой степени, проходящим через середину отрезка --- точку $\overline{x_i}=(x_{i-1}+x_i)/2$, получим формулу прямоугольников:

$$
\int\limits_{a}^bf(x)dx \approx \sum\limits_{i=1}^Nh_if(\frac{x_{i-1}+x_i}{2})
$$

В случае таблично заданных функций удобно в качестве узлов интерполяции выбрать начало и конец отрезка интегрирования, т.е. заменить функцию $f(x)$ многочленом Лагранжа первой степени. В результате получим формулу трапеций:

$$
\int\limits_{a}^bf(x)dx \approx \frac{1}{2}\sum\limits_{i=1}^N(f_i+f_{i-1})h_i
$$

Для повышения порядка точности формулы численного интегрирования заменим подынтегральную кривую параболой --- интерполяционным многочленом второй степени, выбрав в качестве узлов интерполяции концы и середину отрезка интегрирования: $x_{i-1}, x_{i-\frac{1}{2}}=(x_{i-1}+x_i)/2, x_i$\\

Для случая $h_i=\frac{x_i-x_{i-1}}{2}$, получим формулу Симпсона (парабол):

$$
\int\limits_{a}^bf(x)dx \approx \frac{1}{3}\sum\limits_{i=1}^N(f_{i-1}+4f_{i-\frac{1}{2}}+f_i)h_i
$$

В случае интегрирования с постоянным шагом формулы принимают следующий вид:\\

Метод прямоугольников

$$
F=h[y(\frac{x_0+x_1}{2})+y(\frac{x_1+x_2}{2})+...+y(\frac{x_{N-1}+x_N}{2})]
$$

Метод Трапеций

$$
F=h[\frac{y_0}{2}+y_1+y_2+...+y_{N-1}+\frac{y_N}{2}]
$$

Метод Симпсона

$$
F=\frac{h}{3}[y_0+4y_1+2y_2+4y_3+2y_4+...+2y_{N-2}+4y_{N-1}+y_N]
$$\\

\textbf{Реализация}
\begin{lstlisting}
#include <iostream>
#include <fstream>
#include <functional>
#include <cmath>

using namespace std;

int main(int argc, char* argv[]) {
	if (argc != 2) {
		cerr << "Error: arg is incorrect" << endl;
		exit(-1);
	}

	string dataFile = argv[1];
	double h1, h2, X1, Xn;
	
	ifstream in(dataFile, ios::in);
	in >> X1 >> Xn >> h1 >> h2;
	in.close();

	auto Y = [](double x) -> double {
		return x / pow(3 * x + 4, 3.0);
	};

	auto secondDerivateY = [](double x) -> double {
		return 18 * (3 * x - 4) / pow(3 * x + 4, 5.0);
	};

	auto fourthDerivateY = [](double x) -> double {
		return 3240 * (3 * x - 8) / pow(3  * x + 4, 7.0);
	};

	double M2_h1 = 0.0, M2_h2 = 0.0, M4_h1 = 0.0, M4_h2 = 0.0;

	for (double cur = X1; cur <= Xn; cur += h1) {
		M2_h1 = max(M2_h1, abs(secondDerivateY(cur)));
		M4_h1 = max(M4_h1, abs(fourthDerivateY(cur)));
	}
	for (double cur = X1; cur <= Xn; cur += h2) {
		M2_h2 = max(M2_h2, abs(secondDerivateY(cur)));
		M4_h2 = max(M4_h2, abs(fourthDerivateY(cur)));		
	}

	auto rectangleMethod = [&Y](double startX, double endX, double step) -> double {		
		double res = 0.0;
		for (double cur = startX + step; cur <= endX; cur += step) {
			res += Y(0.5 * (2 * cur - step));
		}
		return step * res;
	};

//	auto estimateRectangleMethod = [](double startX, double endX, double step, double M) -> double {
//		return step * step * M * (endX - startX) / 24;
//	};

	auto trapezoidalMethod = [&Y](double startX, double endX, double step) -> double {
		double res = 0.0;
		for (double cur = startX + step; cur <= endX; cur += step) {
			res += Y(cur) + Y(cur - step);
		}
		return 0.5 * step * res;
	};

//	auto estimateTrapezoidalMethod = [](double startX, double endX, double step, double M) -> double {
//		return step * step * M * (endX - startX) / 12;
//	};

	auto SimpsonMethod = [&Y](double startX, double endX, double step) -> double {
		double res = 0.0;
		res += Y(startX) + Y(endX);
		for (double cur = startX + step; cur < endX; cur += 2 * step) {
			res += 4 * Y(cur);
		}
		for (double cur = startX + 2 * step; cur < endX; cur += 2 * step) {
			res += 2 * Y(cur);
		}
		return step * res / 3;
	};

//	auto estimateSimpsonMethod = [](double startX, double endX, double step, double M) -> double {
//		return (endX - startX) * pow(step, 4.0) * M / 180;
//	};

	double k = max(h1, h2) / min(h1, h2);

	cout << "Rectangle method (h1): " << rectangleMethod(X1, Xn, h1) << endl;
	cout << "Rectangle method (h2): " << rectangleMethod(X1, Xn, h2) << endl;
	cout << "estimate: ";
	if (h1 >= h2) cout << rectangleMethod(X1, Xn, h2) + (rectangleMethod(X1, Xn, h2) - rectangleMethod(X1, Xn, h1)) / (pow(k, 2.0) - 1.0) << endl;
	else cout << rectangleMethod(X1, Xn, h1) + (rectangleMethod(X1, Xn, h1) - rectangleMethod(X1, Xn, h2)) / (pow(k, 2.0) - 1.0) << endl;
	cout << "Trapezoidal method (h1): " << trapezoidalMethod(X1, Xn, h1) << endl;
	cout << "Trapezoidal method (h2): " << trapezoidalMethod(X1, Xn, h2) << endl;
	cout << "estimate: ";	
	if (h1 >= h2) cout << trapezoidalMethod(X1, Xn, h2) + (trapezoidalMethod(X1, Xn, h2) - trapezoidalMethod(X1, Xn, h1)) / (pow(k, 2.0) - 1.0) << endl;
	else cout << trapezoidalMethod(X1, Xn, h1) + (trapezoidalMethod(X1, Xn, h1) - trapezoidalMethod(X1, Xn, h2)) / (pow(k, 2.0) - 1.0) << endl;
	cout << "Simpson method (h1): " << SimpsonMethod(X1, Xn, h1) <<  endl;	
	cout << "Simpson method (h2): " << SimpsonMethod(X1, Xn, h2) <<  endl;	
	cout << "estimate: ";	
	if (h1 >= h2) cout << SimpsonMethod(X1, Xn, h2) + (SimpsonMethod(X1, Xn, h2) - SimpsonMethod(X1, Xn, h1)) / (pow(k, 4.0) - 1.0) << endl;
	else cout << SimpsonMethod(X1, Xn, h1) + (SimpsonMethod(X1, Xn, h1) - SimpsonMethod(X1, Xn, h2)) / (pow(k, 4.0) - 1.0) << endl;	
	return 0;
}
\end{lstlisting}
\vspace{0.5cm}

\textbf{Тестирование}\\

\textbf{Входной файл}
\begin{verbatim}
-1 1 0.5 0.25
\end{verbatim}

\textbf{Выходной файл}
\begin{verbatim}
Rectangle method (h1): -0.0709098
Rectangle method (h2): -0.102439
Runge-Romberg estimate: 0.112949
Trapezoidal method (h1): -0.263769
Trapezoidal method (h2): -0.167339
Runge-Romberg estimate: 0.135196
Simpson method (h1): -0.185511
Simpson method (h2): -0.135196
Runge-Romberg estimate: 0.131842
\end{verbatim}

\pagebreak
